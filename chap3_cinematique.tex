\documentclass[a4paper,11pt]{article}

\begin{document}
\section{Cinématique du point matériel}
    \subsection{Qu'est-ce que c'est ?}
    La cinématique s'intéresse à la description du mouvement d'un corps dans l'espace mais \textbf{sans s'intéresser à la cause} de ce mouvement.
    \\
    \\
    Ici on va considérer que ce corps n'a pas de dimension spatiale et donc qu'\textbf{on peut le réduire à un point}.
    Même si dans la vie de tous les jours on a tendance à associer à un corps un volume qui s'étend dans l'espace,
    cette description va nous être très utile lorsque l'on pourra négliger les dimensions du corps et quand celui-ci n'est pas en rotation autour de lui-même.
    

    \subsection{Mise en situation}
    Maintenant qu'on a défini ce qu'était la cinématique, nous allons voir comment on va décrire le mouvement du corps.
    \\
    \\
    Vous participez à un marathon et sportif comme vous êtes, vous prenez avec vous un appareil mesurant le nombre de kilomètres que vous avez parcouru ainsi qu'un chronomètre afin d'évaluer votre performance.
    \\
    \\
    Admettons que vous avez couru 20 km en 2 heures et que toutes les 30 minutes  


\end{document}