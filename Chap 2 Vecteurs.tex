\documentclass[a4paper,11pt]{article}
\begin{document}
\title{Synthèse de physique, partie mécanique}
\date{2020-2021}
\author{Emre Dalgiç et Gabriel de Maere D'Aertrycke}

\maketitle

\section{Introduction}
\section{Les vecteurs}
	\subsection{Grandeurs scalaires et vectorielles}
		\paragraph{Grandeurs scalaires} : \\Ce sont des grandeurs définies 					entièrement par une valeur numérique et une éventuelle unité, les calculs 		les impliquant appliquent les règles de l'algèbre
			
		\paragraph{Grandeurs vectorielles} :\\ Ce sont des objets mathématiques 			définis par plusieurs valeurs numériques:\\
		-Le module (la longueur)\\
		-Le sens\\
		Les calculs les impliquant appliquent les règles de l'algèbre 						vectorielle.
		
	\subsection{Calculs sur les vecteurs}
	\paragraph{Composantes du vecteur}:\\
	le vecteur $\vec{A}$ dans le plan a deux composantes:
	\\ $A_x = (|\vec{A}|cos\phi_A)$ composante selon l'axe x
	\\ $A_y = (|\vec{A}|sin\phi_A)$ composante selon l'axe y
	\\
	\\le vecteur $\vec{A}$ dans l'espace a trois 	composantes:
	\\ $A_x = (|\vec{A}|sin\theta_Acos\phi_A)$ composante selon l'axe x
	\\ $A_y = (|\vec{A}|sin\theta_Asin\phi_A)$ composante selon l'axe y
	\\ $A_z = (|\vec{A}|cos\theta_A)$ composante selon l'axe z
	
	
	%METTRE SCHEMA
	
% Briefly said, spaces separate words, empty lines separate paragraphs.

\section{cinématique du point matériel}



\end{document}
